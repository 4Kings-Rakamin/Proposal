\newpage

\section{Metodologi Penelitian}
% Jelaskan metodologi penelitian akan menggunakan diagram, isi diagramnya sesuai dengan alur crisp dm 


Metodologi penelitian yang akan digunakan dalam proyek ini adalah Cross Industry Standard Process for Data Mining (CRISP-DM). CRISP-DM adalah sebuah model proses yang terstruktur dan berulang yang terdiri dari enam fase utama, yaitu Business Understanding, Data Understanding, Data Preparation, Modeling, Evaluation, dan Deployment. Setiap fase memiliki tujuan dan aktivitas spesifik yang membantu dalam mengelola project ini secara efektif. Gambar \ref{fig:crisp-dm} menunjukkan diagram alur metodologi yang berlandaskan prinsip CRISP-DM yang akan diikuti dalam penelitian ini.

\begin{figure}[H]
    \centering
    \includegraphics[width=0.7\textwidth]{gambar/diagram.png}
    \caption{Diagram Alur Metodologi}
    \label{fig:crisp-dm}
\end{figure}

% Proyek ini menggunakan metodologi CRISP-DM, yaitu pendekatan bertahap dan berfokus pada kebutuhan bisnis untuk proyek data science.
% 1.	Business Understanding: Memahami tujuan bisnis, lalu ubah menjadi tujuan data science.

% 2.	Data Understanding: Mempelajari data yang tersedia untuk melihat kualitas, pola, dan potensi masalah.

% 3.	Data Preparation: Membersihkan dan siapkan data agar bisa dipakai untuk modeling.

% 4.	Modeling: Membuat dan latih model machine learning untuk menemukan pola dan membuat prediksi.

% 5.	Evaluation: Evaluasi kinerja model dan pastikan hasilnya sesuai kebutuhan bisnis.

% 6.	Deployment: Model yang sudah oke diintegrasikan ke sistem bisnis dan dipantau performanya di dunia nyata.
% penjelasan dari diagram
Setiap fase dalam metodologi CRISP-DM memiliki peran penting dalam memastikan keberhasilan proyek ini. Fase Business Understanding bertujuan untuk memahami konteks bisnis dan mengidentifikasi tujuan yang ingin dicapai melalui analisis data. Fase Data Understanding melibatkan eksplorasi awal terhadap dataset untuk menilai kualitas data, mengidentifikasi pola, dan mendeteksi potensi masalah seperti data yang hilang atau outlier.

Fase Data Preparation fokus pada pembersihan dan transformasi data agar siap digunakan dalam proses pemodelan. Ini termasuk penanganan data yang hilang, normalisasi, dan encoding variabel kategorikal. Fase Modeling adalah tahap di mana berbagai algoritma machine learning diterapkan untuk membangun model prediktif berdasarkan data yang telah dipersiapkan. 

Fase Evaluation melibatkan penilaian kinerja model menggunakan metrik yang relevan untuk memastikan bahwa model memenuhi kebutuhan bisnis yang telah ditetapkan. Terakhir, fase Deployment adalah tahap di mana model yang telah dievaluasi dan disetujui diintegrasikan ke dalam sistem bisnis yang ada, serta dipantau secara berkelanjutan untuk memastikan performa yang optimal di lingkungan nyata.

\subsection{Peran Tim dalam Metodologi}

Data analyst berperan penting dalam setiap fase pertama, mulai dari memahami kebutuhan bisnis, melakukan eksplorasi data, menyiapkan data untuk analisis, membangun dan mengevaluasi model, hingga memastikan bahwa model yang dihasilkan dapat diimplementasikan secara efektif dalam konteks bisnis.

Data scientist akan lebih fokus pada fase Modeling dan PreProcessing, di mana mereka akan menerapkan teknik machine learning yang lebih kompleks, melakukan tuning hyperparameter, serta mengevaluasi model dengan metrik yang lebih mendalam untuk memastikan bahwa model tidak hanya akurat tetapi juga dapat diinterpretasikan dan diandalkan.

Data engineer akan berperan utama dalam fase Deployment, di mana mereka akan memastikan bahwa model yang telah dikembangkan dapat diintegrasikan dengan lancar ke dalam infrastruktur teknologi yang ada. Mereka juga akan bertanggung jawab untuk membangun pipeline data yang efisien, mengelola penyimpanan data, serta memastikan bahwa sistem dapat menangani beban kerja yang diperlukan untuk menjalankan model secara real-time atau batch processing sesuai kebutuhan bisnis.

\subsection{TimeLine Project}
Berikut adalah rincian timeline proyek yang direncanakan untuk setiap fase dalam metodologi CRISP-DM, beserta estimasi waktu yang dibutuhkan untuk menyelesaikan masing-masing fase. Tabel \ref{tb:timeline} merangkum jadwal proyek secara keseluruhan.

\begin{table}[H]
    \centering
    \caption{Timeline Project}
    \label{tb:timeline}
    \begin{tabular}{|l|llllll|}
    \hline
    \rowcolor[HTML]{EFEFEF} 
    \multicolumn{1}{|c|}{\cellcolor[HTML]{EFEFEF}}                                   & \multicolumn{1}{c|}{\cellcolor[HTML]{EFEFEF}Aug} & \multicolumn{4}{c|}{\cellcolor[HTML]{EFEFEF}September}                                                                                                                                                & \multicolumn{1}{c|}{\cellcolor[HTML]{EFEFEF}Oct} \\ \cline{2-7} 
    \rowcolor[HTML]{EFEFEF} 
    \multicolumn{1}{|c|}{\multirow{-2}{*}{\cellcolor[HTML]{EFEFEF}Milestone}}        & \multicolumn{1}{c|}{\cellcolor[HTML]{EFEFEF}W4}  & \multicolumn{1}{c|}{\cellcolor[HTML]{EFEFEF}W1} & \multicolumn{1}{c|}{\cellcolor[HTML]{EFEFEF}W2} & \multicolumn{1}{c|}{\cellcolor[HTML]{EFEFEF}W3} & \multicolumn{1}{c|}{\cellcolor[HTML]{EFEFEF}W4} & \multicolumn{1}{c|}{\cellcolor[HTML]{EFEFEF}W1}  \\ \hline
    \begin{tabular}[c]{@{}l@{}}Project Initiation \& \\ Problem Framing\end{tabular} & \multicolumn{2}{l|}{PM \& DA}                                                                      &                                                 &                                                 &                                                 &                                                  \\ \cline{1-3}
    \begin{tabular}[c]{@{}l@{}}Data Acquisition \& \\ Preparation\end{tabular}       & \multicolumn{1}{l|}{}                            & \multicolumn{1}{l|}{DA \& DS}                   &                                                 &                                                 &                                                 &                                                  \\ \cline{1-1} \cline{3-4}
    \begin{tabular}[c]{@{}l@{}}Model Development \\ \& Experimentation\end{tabular}  &                                                  & \multicolumn{1}{l|}{}                           & \multicolumn{1}{l|}{DA \& DS}                   &                                                 &                                                 &                                                  \\ \cline{1-1} \cline{4-5}
    \begin{tabular}[c]{@{}l@{}}Model Evaluation \\ \& Interpretability\end{tabular}  &                                                  &                                                 & \multicolumn{1}{l|}{}                           & \multicolumn{1}{l|}{DS \& BA}                   &                                                 &                                                  \\ \cline{1-1} \cline{5-6}
    \begin{tabular}[c]{@{}l@{}}Deployment \& \\ Business Integration\end{tabular}    &                                                  &                                                 &                                                 & \multicolumn{1}{l|}{}                           & \multicolumn{1}{l|}{DS \& DE}                   &                                                  \\ \cline{1-1} \cline{6-7} 
    Final Presentation                                                               &                                                  &                                                 &                                                 &                                                 & \multicolumn{1}{l|}{}                           & All Role                                         \\ \hline
    \end{tabular}
\end{table}

Agar lebih jelas lagi, berikut adalah penjabaran dari setiap fase beserta estimasi waktu yang dibutuhkan untuk per stage sesuai jadwal yang diberikan Rakamin Academy. Gambar \ref{fig:timeline0} menunjukkan Timeline Stage0 Project secara keseluruhan.

\begin{figure}[H]
    \centering
    \includegraphics[width=0.9\textwidth]{gambar/timeline0.png}
    \caption{Timeline Stage0 Project}
    \label{fig:timeline0}
\end{figure}

Timeline Stage 0 Project dimulai pada minggu ke-4 bulan Agustus dengan dengan detail seperti pada Gambar \ref{fig:timeline0}. Dibuat juga kolom progress task untuk menandai progress mana yang belum dikerjakan, sedang dikerjakan dan belum mulai dikerjakan. Gambar \ref{fig:timeline1} menunjukkan Timeline Stage1 Project dengan progress task.

\begin{figure}[H]
    \centering
    \includegraphics[width=0.9\textwidth]{gambar/timeline1.png}
    \caption{Timeline Stage1 Project dengan Progress Task}
    \label{fig:timeline1}
\end{figure}

Gambar \ref{fig:timeline2} menunjukkan Timeline Stage2 Project dengan progress task.

\begin{figure}[H]
    \centering
    \includegraphics[width=0.9\textwidth]{gambar/timeline2.png}
    \caption{Timeline Stage2 Project dengan Progress Task}
    \label{fig:timeline2}
\end{figure}

Gambar \ref{fig:timeline3} menunjukkan Timeline Stage3 Project dengan progress task.

\begin{figure}[H]
    \centering
    \includegraphics[width=0.9\textwidth]{gambar/timeline3.png}
    \caption{Timeline Stage3 Project dengan Progress Task}
    \label{fig:timeline3}
\end{figure}

Gambar \ref{fig:timeline4} menunjukkan Timeline Stage4 Project dengan progress task.

\begin{figure}[H]
    \centering
    \includegraphics[width=0.9\textwidth]{gambar/timeline4.png}
    \caption{Timeline Stage4 Project dengan Progress Task}
    \label{fig:timeline4}
\end{figure}



\subsection{Risk \& Feasibility Analysis}
Dalam menjalankan proyek ini, terdapat beberapa risiko yang perlu diidentifikasi dan dianalisis untuk memastikan kelancaran proses project. Diharapkan dengan memahami potensi risiko yang ada, tim dapat merancang strategi mitigasi yang efektif guna mengurangi dampak negatif yang mungkin timbul. Tabel \ref{tb:risk-feasibility} merangkum berbagai aspek risiko, potensi risiko yang mungkin dihadapi, strategi mitigasi yang dapat diterapkan, serta penilaian kelayakan dari masing-masing risiko tersebut.



\begin{table}[H]
    \caption{Risk-Feasibility Analysis}
    \label{tb:risk-feasibility}
    \begin{tabular}{|l|l|l|l|}
    \hline
    \rowcolor[HTML]{EFEFEF} 
    \multicolumn{1}{|c|}{\cellcolor[HTML]{EFEFEF}Aspek} & \multicolumn{1}{c|}{\cellcolor[HTML]{EFEFEF}Potensi Risiko}                                                                                                & \multicolumn{1}{c|}{\cellcolor[HTML]{EFEFEF}Strategi Mitigasi}                                                                        & \multicolumn{1}{c|}{\cellcolor[HTML]{EFEFEF}Kelayakan}                                          \\ \hline
    Data                                                & \begin{tabular}[c]{@{}l@{}}Data kandidat bisa \\ tidak lengkap,\\  tidak seimbang \\ (imbalanced), atau\\  mengandung bias \\ (gender, usia).\end{tabular} & \begin{tabular}[c]{@{}l@{}}Lakukan preprocessing, \\ balancing data, feature \\ engineering, serta audit \\ fairness.\end{tabular}    & \begin{tabular}[c]{@{}l@{}}Layak jika dilakukan \\ data cleaning \& \\ monitoring.\end{tabular} \\ \hline
    Teknis                                              & \begin{tabular}[c]{@{}l@{}}Model bisa overfitting\\  atau performa rendah di\\  data baru.\end{tabular}                                                    & \begin{tabular}[c]{@{}l@{}}Gunakan \\ cross-validation, \\ regularisasi, dan \\ retraining berkala.\end{tabular}                      & \begin{tabular}[c]{@{}l@{}}Layak dengan \\ pipeline validasi\\  yang baik.\end{tabular}         \\ \hline
    Operasional                                         & \begin{tabular}[c]{@{}l@{}}HR sulit mengadopsi\\  sistem baru, lebih\\  percaya screening\\  manual.\end{tabular}                                          & \begin{tabular}[c]{@{}l@{}}Berikan pelatihan, buat \\ antarmuka \\ user-friendly, dan \\ jelaskan transparansi \\ model.\end{tabular} & \begin{tabular}[c]{@{}l@{}}Layak jika ada \\ kolaborasi \\ dengan HR.\end{tabular}              \\ \hline
    Etika \& Regulasi                                   & \begin{tabular}[c]{@{}l@{}}Risiko diskriminasi\\  dalam keputusan \\ perekrutan (misalnya \\ gender bias).\end{tabular}                                    & \begin{tabular}[c]{@{}l@{}}Terapkan fairness \\ metrics, hindari variabel \\ sensitif sebagai faktor \\ utama.\end{tabular}           & \begin{tabular}[c]{@{}l@{}}Layak dengan \\ pengawasan \\ etis \& regulasi.\end{tabular}         \\ \hline
    Ekonomi                                             & \begin{tabular}[c]{@{}l@{}}Biaya implementasi dan\\  maintenance model \\ cukup tinggi.\end{tabular}                                                       & \begin{tabular}[c]{@{}l@{}}Bandingkan cost vs \\ benefit (efisiensi waktu, \\ cost per hire, kualitas \\ kandidat).\end{tabular}      & \begin{tabular}[c]{@{}l@{}}Layak jika \\ ROI positif \\ dalam 1–2 tahun.\end{tabular}           \\ \hline
    Keberlanjutan                                       & \begin{tabular}[c]{@{}l@{}}Model bisa usang\\ (model drift) seiring \\ perubahan tren pasar\\  tenaga kerja.\end{tabular}                                  & \begin{tabular}[c]{@{}l@{}}Monitoring performa \\ model, retraining \\ dengan data terbaru \\ setiap periode tertentu.\end{tabular}   & \begin{tabular}[c]{@{}l@{}}Layak dengan \\ komitmen \\ maintenance \\ rutin.\end{tabular}       \\ \hline
    \end{tabular}
    \end{table}

Dengan melakukan analisis risiko ini, tim proyek dapat lebih siap dalam menghadapi tantangan yang mungkin muncul selama pelaksanaan proyek. Setiap risiko yang diidentifikasi telah diberikan strategi mitigasi yang spesifik, sehingga dapat diatasi dengan cara yang paling efektif. Selain itu, penilaian kelayakan dari setiap risiko membantu dalam menentukan prioritas tindakan yang perlu diambil untuk memastikan keberhasilan proyek secara keseluruhan.

\subsection{Penjelasan Dataset}
Dataset yang digunakan dalam proyek ini adalah \texttt{recruitment\_data.csv}, berisi informasi kandidat dan faktor yang dipertimbangkan dalam proses perekrutan. Tujuan pemodelan adalah memprediksi keputusan perekrutan (\textit{HiringDecision}) berdasarkan atribut kandidat.

\subsubsection{Ringkasan Dataset}
\begin{itemize}
    \item \textbf{Jumlah rekaman (baris):} 1{,}500
    \item \textbf{Jumlah fitur (prediktor):} 10
    \item \textbf{Target:} \texttt{HiringDecision} (biner: 0 = tidak diterima, 1 = diterima)
    \item \textbf{Sifat data:} Sintetis (dibuat untuk tujuan pendidikan/proyek data sains)
\end{itemize}

\subsubsection{Definisi Variabel}
Berikut fitur dan target yang tersedia, beserta tipe data, rentang/kategori, dan keterangan singkat. Tabel \ref{tb:var-def} merangkum definisi variabel dalam dataset.

\begin{table}[H]
    \centering
    \caption{Definisi Variabel Dataset}
    \label{tb:var-def}
    \begin{tabular}{|l|l|l|l|}
    \hline
    \rowcolor[HTML]{EFEFEF} 
    Nama Fitur          & Tipe        & Rentang & Keterangan                                                                                      \\ \hline
    Age                 & Numerik     & 20-50   & Umur kandidat                                                                                   \\ \hline
    Gender              & Kategorikal     & 0/1     & 0 = Laki-laki, 1 = Perempuan                                                                    \\ \hline
    EducationLevel      & Kategorikal & 1/2/3/4 & \begin{tabular}[c]{@{}l@{}}1 = S1 (Tipe 1), 2 = S1 (Tipe 2),\\  3 = S2, 4 = S3/PhD\end{tabular} \\ \hline
    ExperienceYears     & Numerik     & 0-15    & Lama pengalaman kerja (tahun)                                                                   \\ \hline
    PreviousCompanies   & Numerik     & 1-5     & \begin{tabular}[c]{@{}l@{}}Jumlah perusahaan tempat bekerja\\  sebelumnya\end{tabular}          \\ \hline
    DistanceFromCompany & Numerik     & 1-50    & Jarak dari rumah ke perusahaan                                                                  \\ \hline
    InterviewScore      & Numerik     & 0-100   & Skor hasil wawancara                                                                            \\ \hline
    SkillScore          & Numerik     & 0-100   & Skor keterampilan teknis                                                                        \\ \hline
    PersonalityScore    & Numerik     & 0-100   & Skor aspek kepribadian                                                                          \\ \hline
    RecruitmentStrategy & Kategorikal & 1/2/3   & \begin{tabular}[c]{@{}l@{}}1 = Agresif, 2 = Moderat, \\ 3 = Konservatif\end{tabular}            \\ \hline
    HiringDecision      & Target      & 0/1     & \begin{tabular}[c]{@{}l@{}}Target: 0 = tidak diterima,\\  1 = diterima\end{tabular}             \\ \hline
    \end{tabular}
    \end{table}

\subsubsection{Catatan Kodefikasi dan Pra-pemrosesan}
\begin{itemize}
    \item \textbf{Gender}: dikodekan sebagai 0 (Laki-laki) dan 1 (Perempuan).
    \item \textbf{EducationLevel}: ordinal 1--4 dengan pemetaan spesifik (S1 Tipe 1, S1 Tipe 2, S2, S3/PhD). Jika korelasi kuat, dapat di \textit{one-hot}.
    \item \textbf{RecruitmentStrategy}: kategorikal 1--3. Umumnya di-\textit{one-hot} untuk model linear; model pohon dapat menggunakan kode numeriknya langsung.
    \item \textbf{Skor (Interview/Skill/Personality)}: berada pada skala 0--100; pertimbangkan penskalaan (\textit{standardization/min-max}) untuk model sensitif skala.
    \item \textbf{Fitur numerik lain} (\texttt{Age}, \texttt{ExperienceYears}, \texttt{PreviousCompanies}, \texttt{DistanceFromCompany}): periksa outlier, distribusi, dan lakukan transformasi/penanganan jika diperlukan.
\end{itemize}

\subsubsection{Sumber dan Lisensi}
Dataset ini dibagikan oleh \textbf{Rabie El Kharoua} dengan lisensi \textbf{CC BY 4.0}. Dataset bersifat \textit{exclusive synthetic} dan ditujukan untuk keperluan edukasi/proyek data sains. Penggunaan diperbolehkan dengan mencantumkan atribusi yang tepat kepada pemilik dataset. DOI dan rincian penyedia data tercantum pada kartu data sumbernya. \parencite{rabie_el_kharoua_2024}

\subsection{EDA (Exploratory Data Analysis)}
EDA adalah langkah awal yang penting dalam analisis data untuk memahami struktur, pola, dan karakteristik dataset. Gambar \ref{fig:eda1} menunjukkan flowchart EDA yang akan dilakukan dalam proyek ini.

\begin{figure}[H]
    \centering
    \includegraphics[width=0.6\textwidth]{gambar/flowchartEDA.png}
    \caption{Flowchart EDA}
    \label{fig:eda1}
\end{figure}

Dengan mengikut langkah EDA yang terstruktur, tim dapat memperoleh wawasan yang mendalam tentang dataset, mengidentifikasi potensi masalah, dan menyiapkan data dengan baik untuk tahap pemodelan selanjutnya. EDA membantu memastikan bahwa model yang dibangun didasarkan pada pemahaman yang kuat tentang data, sehingga meningkatkan peluang keberhasilan proyek secara keseluruhan.

\subsubsection{Handle Tipe Data, NaN, \& Duplikasi}
Data wrangling adalah proses penting dalam persiapan data untuk analisis dan pemodelan. Proses ini melibatkan beberapa langkah kunci yang bertujuan untuk membersihkan, mengubah, dan mengorganisir data agar siap digunakan. Dengan menggunakan df.info(), kita dapat memperoleh gambaran umum tentang struktur dataset, termasuk jumlah entri, tipe data setiap kolom, dan informasi tentang nilai yang hilang. Berikut adalah hasil dari df.info() pada dataset yang digunakan dalam proyek ini.

\begin{center}
    \begin{lstlisting}[language=Python, caption=Info Dataset]
        <class 'pandas.core.frame.DataFrame'>
        RangeIndex: 1500 entries, 0 to 1499
        Data columns (total 11 columns):
         #   Column               Non-Null Count  Dtype  
        ---  ------               --------------  -----  
         0   Age                  1500 non-null   int64  
         1   Gender               1500 non-null   int64  
         2   EducationLevel       1500 non-null   int64  
         3   ExperienceYears      1500 non-null   int64  
         4   PreviousCompanies    1500 non-null   int64  
         5   DistanceFromCompany  1500 non-null   float64
         6   InterviewScore       1500 non-null   int64  
         7   SkillScore           1500 non-null   int64  
         8   PersonalityScore     1500 non-null   int64  
         9   RecruitmentStrategy  1500 non-null   int64  
         10  HiringDecision       1500 non-null   int64  
        dtypes: float64(1), int64(10)
        memory usage: 129.0 KB
    \end{lstlisting}
\end{center}

Dari hasil df.info(), kita dapat melihat bahwa dataset terdiri dari 1500 entri dengan 11 kolom. Semua kolom memiliki tipe data numerik (int64 dan float64), dan tidak ada nilai yang hilang (non-null count sama dengan total entries untuk setiap kolom). Ini menunjukkan bahwa dataset sudah cukup bersih dari segi kelengkapan data, namun masih perlu dilakukan pemeriksaan lebih lanjut terhadap distribusi nilai, outlier, dan potensi inkonsistensi lainnya. Walaupun beberapa fitur numerik memiliki makna kategorikal seperti gender, education level, dan recruitment strategy, hal tersebut tidaklah menjadi masalah karena jika dia bertipe object pada akhirnya akan diubah menjadi numerik juga.

Selanjutnya mengecek apakah ada nilai duplikasi pada dataset. Dengan menggunakan df.duplicated().sum(), kita dapat menghitung jumlah baris yang duplikat dalam dataset. Berikut adalah hasil dari pengecekan duplikasi pada dataset yang digunakan dalam proyek ini.

\newpage
\begin{center}
    \begin{lstlisting}[language=Python, caption=Cek Duplikasi Dataset]
        df.duplicated().sum()

        #output
        np.int64(0)
    \end{lstlisting}
\end{center}

Dari hasil pengecekan duplikasi, kita dapat melihat bahwa tidak ada baris yang duplikat dalam dataset (jumlah duplikasi adalah 0). Ini menunjukkan bahwa setiap entri dalam dataset adalah unik, yang merupakan kondisi ideal untuk analisis data dan pemodelan. Dengan tidak adanya duplikasi, kita dapat melanjutkan ke tahap berikutnya dalam proses data wrangling dengan keyakinan bahwa data yang kita miliki sudah bersih dari masalah duplikasi.

\subsubsection{Analisis Fitur Numerik}
Fitur numerik dalam dataset ini meliputi \texttt{Age}, \texttt{ExperienceYears}, \texttt{PreviousCompanies}, \texttt{DistanceFromCompany}, \texttt{InterviewScore}, \texttt{SkillScore}, dan \texttt{PersonalityScore}. Untuk memahami karakteristik dari fitur-fitur ini, kita dapat melakukan analisis statistik deskriptif dan visualisasi distribusi data. Tabel \ref{tb:statistik-deskriptif} merangkum statistik deskriptif dari fitur numerik dalam dataset.

% Age	Gender	EducationLevel	ExperienceYears	PreviousCompanies	DistanceFromCompany	InterviewScore	SkillScore	PersonalityScore	RecruitmentStrategy	HiringDecision
% count	1500.000000	1500.000000	1500.000000	1500.000000	1500.00000	1500.000000	1500.000000	1500.000000	1500.000000	1500.000000	1500.000000
% mean	35.148667	0.492000	2.188000	7.694000	3.00200	25.505379	50.564000	51.116000	49.387333	1.893333	0.310000
% std	9.252728	0.500103	0.862449	4.641414	1.41067	14.567151	28.626215	29.353563	29.353201	0.689642	0.462647
% min	20.000000	0.000000	1.000000	0.000000	1.00000	1.031376	0.000000	0.000000	0.000000	1.000000	0.000000
% 25%	27.000000	0.000000	2.000000	4.000000	2.00000	12.838851	25.000000	25.750000	23.000000	1.000000	0.000000
% 50%	35.000000	0.000000	2.000000	8.000000	3.00000	25.502239	52.000000	53.000000	49.000000	2.000000	0.000000
% 75%	43.000000	1.000000	3.000000	12.000000	4.00000	37.737996	75.000000	76.000000	76.000000	2.000000	1.000000
% max	50.000000	1.000000	4.000000	15.000000	5.00000	50.992462	100.000000	100.000000	100.000000	3.000000	1.000000


% \begin{table}[H]
%     \centering
%     \caption{Deskripsi Statistik Dataset Kandidat}
%     \begin{tabular}{lcccccccccc}
%     \hline
%     Statistik & Age & Gender & Education & Experience & Prev. Comp & Distance & Interview & Skill & Personality & Recruit. Strategy & Hiring Decision \\
%     \hline
%     count & 1500 & 1500 & 1500 & 1500 & 1500 & 1500 & 1500 & 1500 & 1500 & 1500 & 1500 \\
%     mean  & 35.15 & 0.49 & 2.19 & 7.69 & 3.00 & 25.51 & 50.56 & 51.12 & 49.39 & 1.89 & 0.31 \\
%     std   & 9.25 & 0.50 & 0.86 & 4.64 & 1.41 & 14.57 & 28.63 & 29.35 & 29.35 & 0.69 & 0.46 \\
%     min   & 20.00 & 0.00 & 1.00 & 0.00 & 1.00 & 1.00 & 0.00 & 0.00 & 0.00 & 1.00 & 0.00 \\
%     25\%  & 27.00 & 0.00 & 2.00 & 4.00 & 2.00 & 12.84 & 25.00 & 25.75 & 23.00 & 1.00 & 0.00 \\
%     50\%  & 35.00 & 0.00 & 2.00 & 8.00 & 3.00 & 25.50 & 52.00 & 53.00 & 49.00 & 2.00 & 0.00 \\
%     75\%  & 43.00 & 1.00 & 3.00 & 12.00 & 4.00 & 37.74 & 75.00 & 76.00 & 76.00 & 2.00 & 1.00 \\
%     max   & 50.00 & 1.00 & 4.00 & 15.00 & 5.00 & 50.99 & 100.00 & 100.00 & 100.00 & 3.00 & 1.00 \\
%     \hline
%     \end{tabular}
%     \end{table}
    
% \begin{table}[H]
%     \centering
%     \caption{Deskripsi Statistik Dataset Kandidat}
%     \begin{tabular}{lccccccccccc}
%     \hline
%     Statistik & Age & Gender & Education & Experience & Prev. Comp & Distance & Interview & Skill & Personality & Recruit. Strategy & Hiring Decision \\
%     \hline
%     count & 1500 & 1500 & 1500 & 1500 & 1500 & 1500 & 1500 & 1500 & 1500 & 1500 & 1500 \\
%     mean  & 35.15 & 0.49 & 2.19 & 7.69 & 3.00 & 25.51 & 50.56 & 51.12 & 49.39 & 1.89 & 0.31 \\
%     std   & 9.25 & 0.50 & 0.86 & 4.64 & 1.41 & 14.57 & 28.63 & 29.35 & 29.35 & 0.69 & 0.46 \\
%     min   & 20.00 & 0.00 & 1.00 & 0.00 & 1.00 & 1.00 & 0.00 & 0.00 & 0.00 & 1.00 & 0.00 \\
%     25\%  & 27.00 & 0.00 & 2.00 & 4.00 & 2.00 & 12.84 & 25.00 & 25.75 & 23.00 & 1.00 & 0.00 \\
%     50\%  & 35.00 & 0.00 & 2.00 & 8.00 & 3.00 & 25.50 & 52.00 & 53.00 & 49.00 & 2.00 & 0.00 \\
%     75\%  & 43.00 & 1.00 & 3.00 & 12.00 & 4.00 & 37.74 & 75.00 & 76.00 & 76.00 & 2.00 & 1.00 \\
%     max   & 50.00 & 1.00 & 4.00 & 15.00 & 5.00 & 50.99 & 100.00 & 100.00 & 100.00 & 3.00 & 1.00 \\
%     \hline
%     \end{tabular}
%     \end{table}

\begin{table}[H]
    \centering
    \caption{Statistik Deskriptif Fitur Numerik}
    \label{tb:statistik-deskriptif}
    \begin{tabular}{|l|l|l|l|l|l|l|l|l|}
    \hline
    \rowcolor[HTML]{EFEFEF} 
    Fitur               & Count & Mean  & Std   & Min   & 25\%  & 50\%  & 75\%  & Max   \\ \hline
    Age                 & 1500  & 35.15 & 9.25  & 20.00 & 27.00 & 35.00 & 43.00 & 50.00 \\ \hline
    ExperienceYears     & 1500  & 7.69  & 4.64  & 0.00  & 4.00  & 8.00  & 12.00 & 15.00 \\ \hline
    PreviousCompanies   & 1500  & 3.00  & 1.41  & 1.00  & 2.00  & 3.00  & 4.00  & 5.00  \\ \hline
    DistanceFromCompany & 1500  & 25.51 & 14.57 & 1.00  & 12.84 & 25.50 & 37.74 & 50.99 \\ \hline
    InterviewScore      & 1500  & 50.56 & 28.63 & 0.00  & 25.00 & 52.00 & 75.00 & 100.00\\ \hline
    SkillScore          & 1500  & 51.12 & 29.35 & 0.00  & 25.75 & 53.00 & 76.00 & 100.00\\ \hline
    PersonalityScore    & 1500  & 49.39 & 29.35 & 0.00   &23.00&49.00&76.00&100.00\\ \hline
    \end{tabular}
\end{table}

    


