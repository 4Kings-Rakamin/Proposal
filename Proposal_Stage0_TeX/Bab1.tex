\section{Pendahuluan}

Dalam perekrutan karyawan, pengambilan keputusan yang tepat, cepat, efektif serta efisien sangat penting untuk memastikan bahwa perusahaan mendapatkan kandidat yang paling sesuai dengan kebutuhan dan budaya organisasi. Dengan kemajuan teknologi dan analisis data, perusahaan kini dapat memanfaatkan data historis yang diolah menjadi model prediksi untuk meningkatkan proses perekrutan mereka lebih baik lagi. 

Project ini disusun dengan pendekatan data-driven strategy menggunakan machine learning untuk membantu perusahaan memprediksi pelamar yang potensial. Dengan model prediksi ini, perusahaan dapat melakukan pengambilan keputusan secara cepat dan tepat tanpa adanya bias subjektif.


% `Berikut merupakan List fitur :`
% 1. Age : Umur kandidat
% 2. Gender : Jenis kelamin kandidat (0 , 1) 
% 3. EducationLevel : Tingkat pendidikan kandidat (1 - 5)
% 4. ExperienceYears : Jumlah tahun pengalaman kandidat
% 5. PreviousCompanies : Jumlah perusahaan sebelumnya kandidat
% 6. DistanceFromCompany : Jarak dari rumah ke perusahaan 
% 7. InterviewScore : Skor wawancara kandidat (1 - 100)
% 8. SkillScore : Skor keterampilan kandidat (1 - 100)
% 9. PersonalityScore : Skor kepribadian kandidat (1 - 100)
% 10. RecruitmentStrategy : Strategi rekrutmen (1,2,3)

% Label Target :
% 11. HiringDecision : Keputusan perekrutan (0 = Tidak, 1 = Ya)

%buat latar belakang tentang data yang digunakan
% \subsection{Latar Belakang}
% Dalam dunia perekrutan karyawan, pengambilan keputusan yang tepat sangat penting untuk memastikan bahwa perusahaan mendapatkan kandidat yang paling sesuai dengan kebutuhan dan budaya organisasi. Dengan kemajuan teknologi dan analisis data, perusahaan kini dapat memanfaatkan data historis untuk meningkatkan proses perekrutan mereka. Dataset yang digunakan dalam proyek ini berisi informasi tentang berbagai kandidat yang telah melamar pekerjaan di sebuah perusahaan, termasuk fitur-fitur seperti umur, jenis kelamin, tingkat pendidikan, pengalaman kerja, dan skor wawancara. Dengan menganalisis data ini, kita dapat mengidentifikasi pola dan faktor-faktor yang mempengaruhi keputusan perekrutan, sehingga membantu perusahaan dalam membuat keputusan yang lebih baik di masa depan.

% \subsection{Riset Bisnis}

% Dalam konteks bisnis, proses perekrutan yang efisien dan efektif sangat penting untuk keberhasilan jangka panjang perusahaan. Dengan menggunakan data historis dari proses perekrutan sebelumnya, perusahaan dapat mengidentifikasi karakteristik kandidat yang paling berhasil dan sesuai dengan kebutuhan organisasi. Hal ini tidak hanya membantu dalam mengurangi biaya dan waktu yang dihabiskan untuk proses perekrutan, tetapi juga mening katkan kualitas karyawan yang direkrut. Dengan demikian, analisis data ini dapat memberikan wawasan berharga bagi tim HR dan manajemen dalam mengoptimalkan strategi perekrutan mereka.

% %tekankan untuk manajemen biaya yang efisien menggunakan model machine learning
% Adapun penggunaan model machine learning dalam proses perekrutan ini dapat membantu perusahaan dalam mengelola biaya secara lebih efisien. Dengan memprediksi kandidat yang memiliki kemungkinan besar untuk diterima berdasarkan data historis, perusahaan dapat mengurangi jumlah pengiriman email penolakan. Hal ini tidak hanya menghemat waktu dan sumber daya, tetapi juga memungkinkan tim HR untuk lebih fokus pada aspek-aspek lain dari proses perekrutan, seperti pengembangan karyawan dan retensi. Dengan demikian, implementasi model machine learning dalam proses perekrutan dapat memberikan manfaat ekonomi yang signifikan bagi perusahaan.


\subsection{Problem Statement}

% 1500 kandidat pada dataset, ada 69% kandidat yang diterima dan 31% kandidat yang ditolak
% HiringDecision
% 0    69.0
% 1    31.0
% Name: proportion, dtype: float64

Proses rekrutmen, perusahaan kerap menghadapi berbagai tantangan, mulai dari tingginya jumlah pelamar, keterbatasan waktu untuk melakukan penilaian, hingga adanya subjektivitas yang dapat memengaruhi konsistensi keputusan.

Menurut CareerBuilder, hampir tiga perempat perusahaan yang melakukan rekrutmen yang buruk melaporkan rata-rata kerugian biaya sebesar USD 14.900. Selain itu, 74\% pengusaha menyatakan pernah merekrut orang yang tidak tepat untuk suatu posisi. Kondisi ini sering kali disebabkan oleh human error dalam proses seleksi, terutama karena banyaknya kandidat yang harus disaring serta adanya tenggat waktu yang ketat, sehingga memperbesar peluang terjadinya kesalahan. \parencite{example2025}

Dalam dunia kerja yang serba cepat, kecepatan menjadi salah satu faktor kunci bagi perusahaan untuk mempertahankan keunggulan kompetitif. Namun, masih banyak perusahaan yang menjalankan proses rekrutmen secara manual, sehingga proses perekrutan menjadi kurang efisien. Oleh karena itu, dibutuhkan solusi berbasis kecerdasan buatan (AI) yang mampu mempercepat proses seleksi, mengurangi beban operasional, sekaligus meningkatkan kualitas keputusan rekrutmen.

Kompleksitas parameter penilaian seperti kompetensi, keterampilan, serta faktor demografi semakin menambah risiko perusahaan gagal menemukan kandidat terbaik. Untuk itu, penelitian ini memanfaatkan dataset berisi 1.500 data historis kandidat (recruitment data.csv), yang mencakup informasi usia, gender, tingkat pendidikan, pengalaman kerja, jumlah perusahaan sebelumnya, jarak tempat tinggal dari kantor, skor wawancara, skor keterampilan, skor kepribadian, strategi rekrutmen, hingga keputusan akhir perekrutan (Hiring Decision) dan mengasilkan model prediksi yang akurat.



\subsection{Goals, Objectives, and Business Metrics}

Tujuan utama dari proyek ini adalah membangun model machine learning yang dapat membantu proses pengambilan keputusan perekrutan karyawan secara lebih cepat, objektif, dan konsisten. Model ini diharapkan mampu memberikan rekomendasi kandidat potensial serta mengurangi bias subjektif dalam proses seleksi.

\textbf{Objectives:}
\begin{enumerate}
    \item Mengembangkan model prediktif berdasarkan data historis recruitment untuk mengklasifikasikan kandidat apakah diterima atau tidak
    \item Mempercepat proses identifikasi kandidat dan mengurangi bias subjektif  dalam mengambil keputusan.
    \item Menghindari salah rekrut karyawan yang berpotensi merugikan peruhahaan.
    \item Menurunkan biaya rekrutmen agar lebih efesien
\end{enumerate}

\textbf{Business Metrics:}
\begin{enumerate}
    \item \textbf{Akurasi Model:} Akurasi $\geq 86\%$ untuk memastikan keputusan perekrutan yang lebih tepat.
    \item \textbf{Precision:} Precision $\geq 80\%$, agar recruiter dapat mengurangi jumlah kandidat “false positive” yang tidak sesuai.
    \item \textbf{AUC:} Nilai AUC $\geq 85\%$ untuk menjamin kestabilan performa model pada berbagai threshold dalam proses shortlisting kandidat.
    \item \textbf{Efisiensi Waktu:} Mengurangi waktu screening kandidat sebesar 60\% dalam waktu 3 bulan setelah model prediksi mulai digunakan.
    \item \textbf{Reduksi Biaya:} Menurunkan biaya operasional terkait dengan proses perekrutan sebesar 15\% dalam waktu 3 bulan setelah model prediksi mulai digunakan.
\end{enumerate}

\subsection{Gap Analysis}
Saat ini belum tersedia sistem prediksi keputusan perekrutan berbasis data yang cepat dan objektif. Proses pengambilan keputusan HR masih didominasi oleh penilaian subjektif, membutuhkan waktu yang relatif lama, serta tidak memberikan insight yang jelas terkait faktor-faktor yang memengaruhi keputusan, seperti kompetensi, keterampilan, dan aspek demografi kandidat. Kondisi ini berpotensi menurunkan konsistensi dan efektivitas dalam proses seleksi karyawan baru.

Dengan adanya model machine learning yang diusulkan, diharapkan dapat mengisi gap ini dengan menyediakan alat bantu yang mampu menganalisis data historis secara cepat dan akurat. Selain itu EDA juga dilakukan untuk menemukan aspek aspek apa saja yang berpengaruh besar dalam menghasilkan keputusan. Dimana outputnya yaitu Insight dan Model machine learning ini akan membantu HR dalam mengidentifikasi kandidat potensial berdasarkan fitur-fitur yang relevan, sehingga mengurangi ketergantungan pada penilaian subjektif dan mempercepat proses seleksi. Selain itu, model ini juga akan memberikan wawasan tentang faktor-faktor kunci yang memengaruhi keputusan perekrutan, sehingga perusahaan dapat mengoptimalkan strategi rekrutmen mereka ke depannya.

\subsection{Ideal Condition \& Expected Impact}
Sistem mampu mengidentifikasi kandidat potensial secara cepat, objektif, dan tepat sasaran melalui pemetaan kandidat yang layak direkrut. Dampaknya, proses pengambilan keputusan menjadi lebih efisien sehingga HR dapat memfokuskan waktu pada kandidat yang benar-benar potensial dan sesuai kebutuhan perusahaan.




