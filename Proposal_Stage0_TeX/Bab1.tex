\section{Pendahuluan}

% `Berikut merupakan List fitur :`
% 1. Age : Umur kandidat
% 2. Gender : Jenis kelamin kandidat (0 , 1) 
% 3. EducationLevel : Tingkat pendidikan kandidat (1 - 5)
% 4. ExperienceYears : Jumlah tahun pengalaman kandidat
% 5. PreviousCompanies : Jumlah perusahaan sebelumnya kandidat
% 6. DistanceFromCompany : Jarak dari rumah ke perusahaan 
% 7. InterviewScore : Skor wawancara kandidat (1 - 100)
% 8. SkillScore : Skor keterampilan kandidat (1 - 100)
% 9. PersonalityScore : Skor kepribadian kandidat (1 - 100)
% 10. RecruitmentStrategy : Strategi rekrutmen (1,2,3)

% Label Target :
% 11. HiringDecision : Keputusan perekrutan (0 = Tidak, 1 = Ya)

%buat latar belakang tentang data yang digunakan
\subsection{Latar Belakang}
Dalam dunia perekrutan karyawan, pengambilan keputusan yang tepat sangat penting untuk memastikan bahwa perusahaan mendapatkan kandidat yang paling sesuai dengan kebutuhan dan budaya organisasi. Dengan kemajuan teknologi dan analisis data, perusahaan kini dapat memanfaatkan data historis untuk meningkatkan proses perekrutan mereka. Dataset yang digunakan dalam proyek ini berisi informasi tentang berbagai kandidat yang telah melamar pekerjaan di sebuah perusahaan, termasuk fitur-fitur seperti umur, jenis kelamin, tingkat pendidikan, pengalaman kerja, dan skor wawancara. Dengan menganalisis data ini, kita dapat mengidentifikasi pola dan faktor-faktor yang mempengaruhi keputusan perekrutan, sehingga membantu perusahaan dalam membuat keputusan yang lebih baik di masa depan.

\subsection{Riset Bisnis}

Dalam konteks bisnis, proses perekrutan yang efisien dan efektif sangat penting untuk keberhasilan jangka panjang perusahaan. Dengan menggunakan data historis dari proses perekrutan sebelumnya, perusahaan dapat mengidentifikasi karakteristik kandidat yang paling berhasil dan sesuai dengan kebutuhan organisasi. Hal ini tidak hanya membantu dalam mengurangi biaya dan waktu yang dihabiskan untuk proses perekrutan, tetapi juga mening katkan kualitas karyawan yang direkrut. Dengan demikian, analisis data ini dapat memberikan wawasan berharga bagi tim HR dan manajemen dalam mengoptimalkan strategi perekrutan mereka.

%tekankan untuk manajemen biaya yang efisien menggunakan model machine learning
Adapun penggunaan model machine learning dalam proses perekrutan ini dapat membantu perusahaan dalam mengelola biaya secara lebih efisien. Dengan memprediksi kandidat yang memiliki kemungkinan besar untuk diterima berdasarkan data historis, perusahaan dapat mengurangi jumlah pengiriman email penolakan. Hal ini tidak hanya menghemat waktu dan sumber daya, tetapi juga memungkinkan tim HR untuk lebih fokus pada aspek-aspek lain dari proses perekrutan, seperti pengembangan karyawan dan retensi. Dengan demikian, implementasi model machine learning dalam proses perekrutan dapat memberikan manfaat ekonomi yang signifikan bagi perusahaan.


\subsection{Problem Statement}

% 1500 kandidat pada dataset, ada 69% kandidat yang diterima dan 31% kandidat yang ditolak
% HiringDecision
% 0    69.0
% 1    31.0
% Name: proportion, dtype: float64

Proses rekrutmen sering menghadapi berbagai tantangan, seperti banyaknya jumlah pelamar, keterbatasan waktu dalam melakukan penilaian, serta tingginya tingkat subjektivitas yang dapat memengaruhi konsistensi pengambilan keputusan. Kompleksitas parameter penilaian yang mencakup kompetensi, keterampilan, dan faktor demografi semakin meningkatkan risiko perusahaan gagal merekrut kandidat potensial. Dataset yang digunakan dalam penelitian ini (recruitment\_data.csv) berisi 1.500 data historis kandidat yang mencakup usia, gender, tingkat pendidikan, pengalaman kerja, jumlah perusahaan sebelumnya, jarak tempat tinggal dari kantor, skor wawancara, skor keterampilan, skor kepribadian, strategi rekrutmen, serta keputusan akhir perekrutan (HiringDecision). 

Dari hasil analisis awal, ditemukan ketidakseimbangan yang signifikan dalam distribusi kelas, di mana sekitar 69\%  kandidat diterima dan 31\% ditolak. Kondisi ini menyebabkan perusahaan harus mengirim banyak email penolakan, yang pada akhirnya meningkatkan biaya operasional dan menurunkan efisiensi rekrutmen. Oleh karena itu, proyek ini bertujuan untuk mengembangkan model machine learning yang mampu memprediksi keputusan perekrutan secara akurat sekaligus menangani ketidakseimbangan kelas, sehingga perusahaan dapat mengoptimalkan proses rekrutmen dan meminimalisasi risiko kegagalan dalam merekrut kandidat potensial.


\subsection{Goals, Objectives, and Business Metrics}

Tujuan utama dari proyek ini adalah membangun model machine learning yang dapat membantu proses pengambilan keputusan perekrutan karyawan secara lebih cepat, objektif, dan konsisten. Dengan memanfaatkan data historis perekrutan, model ini diharapkan mampu memberikan rekomendasi kandidat potensial serta mengurangi bias subjektif dalam proses seleksi.

\textbf{Objectives:}
\begin{enumerate}
    \item Mengembangkan model prediktif berdasarkan data historis recruitment untuk mengklasifikasikan kandidat apakah diterima atau tidak.
    \item Mengidentifikasi kandidat potensial yang memiliki probabilitas keberhasilan tinggi.
    \item Mempercepat proses identifikasi kandidat sebagai penilaian awal HR dalam mengambil keputusan.
    \item Mengurangi bias subjektif dengan menyediakan sistem pendukung keputusan berbasis data.
    \item Menyediakan interpretasi hasil model melalui analisis feature importance agar perusahaan mengetahui faktor yang paling memengaruhi keputusan perekrutan.
\end{enumerate}

\textbf{Business Metrics:}
\begin{enumerate}
    \item \textbf{Akurasi Model:} Akurasi prediksi $\geq 85\%$ untuk memastikan keputusan perekrutan yang lebih tepat.
    \item \textbf{Precision:} Precision $\geq 80\%$, agar recruiter dapat mengurangi jumlah kandidat ``false positive'' yang tidak sesuai.
    \item \textbf{Recall dan F1-Score:} Recall yang baik untuk meminimalisasi hilangnya kandidat potensial, dengan target F1-Score $\geq 75\%$ agar model seimbang antara precision dan recall.
    \item \textbf{AUC:} Nilai AUC $\geq 85\%$ untuk menjamin kestabilan performa model pada berbagai threshold dalam proses shortlisting kandidat.
    \item \textbf{Efisiensi Waktu:} Mengurangi waktu screening kandidat sehingga proses rekrutmen lebih cepat dan efisien.
    \item \textbf{Reduksi Biaya:} Menurunkan biaya operasional terkait dengan pengiriman email penolakan dan proses administrasi perekrutan.
\end{enumerate}

\subsection{Gap Analysis}
Saat ini belum tersedia sistem prediksi keputusan perekrutan berbasis data yang cepat dan objektif. Proses pengambilan keputusan HR masih didominasi oleh penilaian subjektif, membutuhkan waktu yang relatif lama, serta tidak memberikan insight yang jelas terkait faktor-faktor yang memengaruhi keputusan, seperti kompetensi, keterampilan, dan aspek demografi kandidat. Kondisi ini berpotensi menurunkan konsistensi dan efektivitas dalam proses seleksi karyawan baru.

Dengan adanya model machine learning yang diusulkan, diharapkan dapat mengisi gap ini dengan menyediakan alat bantu yang mampu menganalisis data historis secara cepat dan akurat. Model ini akan membantu HR dalam mengidentifikasi kandidat potensial berdasarkan fitur-fitur yang relevan, sehingga mengurangi ketergantungan pada penilaian subjektif dan mempercepat proses seleksi. Selain itu, model ini juga akan memberikan wawasan tentang faktor-faktor kunci yang memengaruhi keputusan perekrutan, sehingga perusahaan dapat mengoptimalkan strategi rekrutmen mereka ke depannya.

\subsection{Ideal Condition \& Expected Impact}
Sistem mampu mengidentifikasi kandidat potensial secara cepat, objektif, dan tepat sasaran melalui pemetaan kandidat yang layak direkrut. Dampaknya, proses pengambilan keputusan menjadi lebih efisien sehingga HR dapat memfokuskan waktu pada kandidat yang benar-benar potensial dan sesuai kebutuhan perusahaan.




